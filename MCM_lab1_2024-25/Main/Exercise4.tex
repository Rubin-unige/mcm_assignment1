\section{Exercise 4 - Rotation Matrix to Euler Angles} \label{P4}

In this exercise, the conversion from a rotation matrix to equivalent Yaw-Pitch-Roll (YPR) angles is implemented. Yaw, pitch, and roll angles represent rotations around the z, y, and x axes, respectively. The relationship between the rotation matrix \( R \) and the YPR angles \( (\psi, \theta, \phi) \) is examined.

\subsection{Q 4.1 - Implementation of Rotation Matrix to Yaw-Pitch-Roll Conversion}

The MATLAB function \texttt{RotToYPR} is implemented to convert a given rotation matrix into the corresponding Euler angles using the Yaw-Pitch-Roll (YPR) convention. This function produces three outputs: yaw (\(\psi\)), pitch (\(\theta\)), and roll (\(\phi\)) angles. 

\begin{enumerate}
    \item \textbf{Input Validation:}
    The function first verifies that the input \( R \) is a valid rotation matrix. This involves performing the following checks:
        \begin{itemize}
        \item \textit{Size Check:} Ensures that \( R \) is \( 3 \times 3 \) by verifying that both dimensions equal 3. If this condition is not met, an error is raised.
        
        \item \textit{Orthogonality Check:} Confirms that \( R \) is orthogonal by computing \( R R^T \) and comparing it to the identity matrix \( I \). This ensures that the rows (and columns) of \( R \) are orthonormal vectors.
        
        \item \textit{Determinant Check:} Checks that the determinant of \( R \) is 1, confirming it as a proper rotation matrix.
    \end{itemize}

    \item \textbf{Angle Calculation:}
    After the input validation, the function calculates the Euler angles:
    \begin{itemize}
        \item The pitch angle \(\theta\) is calculated as:
        \begin{equation}
        \theta = -\arcsin(R_{13})
        \end{equation}
        where \( R_{13} \) is the element in the first row and third column of \( R \).

        \item The yaw angle \(\psi\) and roll angle \(\phi\) are conditionally calculated based on the cosine of \(\theta\) to handle potential singularities:
        \begin{itemize}
            \item If \( \cos(\theta) > 1 \times 10^{-6} \) (indicating a non-singular case), yaw and roll are computed as:
            \begin{equation}
            \psi = \arctan2\left(\frac{R_{23}}{\cos(\theta)}, \frac{R_{33}}{\cos(\theta)}\right)
            \end{equation}
            \begin{equation}
            \phi = \arctan2\left(\frac{R_{12}}{\cos(\theta)}, \frac{R_{11}}{\cos(\theta)}\right)
            \end{equation}
            where \( R_{23} \) and \( R_{33} \) are elements in the second and third rows of \( R \), respectively, for yaw, and \( R_{12} \) and \( R_{11} \) are elements in the first row for roll.

            \item If \( \cos(\theta) \) is close to zero (indicating a singularity), yaw is set to 0 (as it becomes indeterminate), and roll is computed as:
            \begin{equation}
            \psi = 0
            \end{equation}
            \begin{equation}
            \phi = \arctan2(R_{21}, R_{22})
            \end{equation}
            where \( R_{21} \) and \( R_{22} \) are elements in the rotation matrix. This approach maintains stability even in singular cases.
        \end{itemize}
    \end{itemize}
\end{enumerate}

\subsection{Q 4.2 - Test Case 1 }

In this test case, the rotation matrix is given as:
\[
R =
\begin{pmatrix}
1 & 0 & 0 \\
0 & 0 & -1 \\
0 & 1 & 0
\end{pmatrix}
\]

The calculated Yaw-Pitch-Roll (YPR) angles are:
\begin{itemize}
    \item Yaw (\(\psi\)): \(-1.5708\) radians
    \item Pitch (\(\theta\)): \(0\) radians
    \item Roll (\(\phi\)): \(0\) radians
\end{itemize}

The yaw angle of \(-1.5708\) radians indicates a 90-degree rotation to the left about the z-axis. The pitch and roll angles being zero signify that the object remains level along the y-axis and upright along the x-axis, respectively.

\subsection{Q 4.3 - Test Case 2}

In this test case, the rotation matrix is given as:
\[
R = \begin{pmatrix}
\frac{1}{2} & -\frac{\sqrt{3}}{2} & 0 \\
\frac{\sqrt{3}}{2} & \frac{1}{2} & 0 \\
0 & 0 & 1
\end{pmatrix}
\]

The calculated Yaw-Pitch-Roll (YPR) angles are:
\begin{itemize}
    \item Yaw (\(\psi\)): \(0\) radians
    \item Pitch (\(\theta\)): \(0\) radians
    \item Roll (\(\phi\)): \(-1.0472\) radians (which is equivalent to \(-\frac{\pi}{3}\))
\end{itemize}

The results indicate that the yaw and pitch angles are both zero, suggesting no rotation around these axes, while the roll angle of \(-1.0472\) radians represents a rotation of \(-\frac{\pi}{3}\) radians about the x-axis.

\subsection{Q 4.4 - Test Case 3}

In this test case, the rotation matrix is defined as:
\[
R = \begin{pmatrix}
0 & -\frac{\sqrt{2}}{2} & \frac{\sqrt{2}}{2} \\
0.5 & \frac{\sqrt{6}}{4} & \frac{\sqrt{6}}{4} \\
-\frac{\sqrt{3}}{2} & \frac{\sqrt{2}}{4} & \frac{\sqrt{2}}{4}
\end{pmatrix}
\]

The calculated Yaw-Pitch-Roll (YPR) angles are:
\begin{itemize}
    \item Yaw (\(\psi\)): \(1.0472\) radians
    \item Pitch (\(\theta\)): \(-0.7854\) radians
    \item Roll (\(\phi\)): \(-1.5708\) radians
\end{itemize}

The yaw angle of \(1.0472\) radians indicates a rotation of approximately \(60^\circ\) around the z-axis, suggesting a rightward turn. The pitch angle of \(-0.7854\) radians corresponds to a downward tilt of about \(-45^\circ\), showing a descending rotation around the y-axis. The roll angle of \(-1.5708\) radians represents a clockwise rotation of approximately \(-90^\circ\) about the x-axis, indicating a significant tilt.
